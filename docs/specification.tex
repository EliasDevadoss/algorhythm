\documentclass[10pt]{article}


% Lines beginning with the percent sign are comments
% This file has been commented to help you understand more about LaTeX


% DO NOT EDIT THE LINES BETWEEN THE TWO LONG HORIZONTAL LINES


%---------------------------------------------------------------------------------------------------------


% Packages add extra functionality.
\usepackage{times,graphicx,epstopdf,fancyhdr,amsfonts,amsthm,amsmath,algorithm,algorithmic,xspace,hyperref}
\usepackage[left=1in,top=1in,right=1in,bottom=1in]{geometry}
\usepackage{sect sty}   %For centering section headings
\usepackage{enumerate}  %Allows more labeling options for enumerate environments
\usepackage{epsfig}
\usepackage[space]{grffile}
\usepackage{booktabs}
\usepackage{forest}
\usepackage[T1]{fontenc}
\usepackage[utf8]{inputenc}
\usepackage{longtable}


% This will set LaTeX to look for figures in the same directory as the .tex file
\graphicspath{.} % The dot means current directory.


\pagestyle{fancy}


\lhead{Final Project}
\rhead{\today}
\lfoot{CSCI 334: Principles of Programming Languages}
\cfoot{\thepage}
\rfoot{Fall 2023}


% Some commands for changing header and footer format
\renewcommand{\headrulewidth}{0.4pt}
\renewcommand{\headwidth}{\textwidth}
\renewcommand{\footrulewidth}{0.4pt}


% These let you use common environments
\newtheorem{claim}{Claim}
\newtheorem{definition}{Definition}
\newtheorem{theorem}{Theorem}
\newtheorem{lemma}{Lemma}
\newtheorem{observation}{Observation}
\newtheorem{question}{Question}


\setlength{\parindent}{0cm}


\begin{document}
 \section*{AlgoRhythm Documentation}


Elias Devadoss, Himal Pandey


\subsection{Introduction}


   Our language solves the problem of having to generate and record a song when you have an idea
you want to implement. Instead, only the pieces to the song need to be given, and the language can
output a fully realized song based on the program. While in the modern day we can have AI generate
songs for us, there is still some difficulty communicating exactly what you want to a chatbot or
similar service, as well as a lack of personal creativity in having a computer generate a song.
On the flip side, being able to think of a song from nothing can be quite challenging and many
do not know where to start.


   This language serves as a jumping board for song ideas, where rhythm and melody can be
mechanically reproduced and realized, without needing instruments on hand or physical skill. This
specific language is also vital, as there are music production services where users can input
exactly what they want to hear and receive an output. However, this language allows for a more
loosely defined “song” to be created where a user only needs a relatively simple collection of
chords, rhythm, and structure to be able to set the background for a song they are creating and
test the waters. Finally, this language lowers the barriers for those who are not musically
talented or inclined by making it relatively simple and easy to make a song, even for those
with no experience or talent in the matter.


\subsection{Design Principles}


This language aims to be one that is simple and neat. The aesthetic will be one that aims for
clarity and neatness. Due to the nature of how many specifics may have to be given, programs
may be somewhat lengthy. However, they will be simple and easy to read. The idea is to have the
language be one that follows the general principles of the language being easy to use and not a
barrier for those who are not musically proficient. Therefore, the language will also be one
that does not require learning much syntax or specifics, so that it is easy to implement.


\subsection{Examples}


i. Twinkle, Twinkle Little Star
\begin{verbatim}
<song> ::= 125 bpm <melody> <beat> <chordList>
<melody> ::= C4 1 beats, C4 1 beats, G4 1 beats, G4 1 beats, A4 1 beats,
            A4 1 beats,G4 2 beats,
<beat> ::= kick 1, snare 3, hi-hat 1 2 3 4
<chordList> ::= C4 E4 G2 2 beats, C4 E4 G2 2 beats, F4 A5 C5 2 beats,
               C4 E4 G2 2 beats,
\end{verbatim}
This will be executed via a command like {\tt dotnet run "example-1.txt"}. The program will
then be read and outputted to a MIDI file, which can be played by the user.


Output: Twinkle, twinkle little star with backing chords of C, C, F, C and a basic beat behind.


\vspace{\baselineskip}


ii. 2:20
\begin{verbatim}
<song> ::= 65 bpm <melody> <beat> <chordList>
<melody> ::= e 2.75 beats, A3 0.25 beats, C4 0.25 beats, D4 0.25 beats,
            F4 0.25 beats, A3 0.25 beats, C4 0.5 beats, D4 0.5 beats,
            e 0.5 beats, D3 1 beats, e 0.25 beats
<beat> ::= kick 1 1.5, snare 3, crash 1, ride 2 3 4
<chordList> ::= e
\end{verbatim}
This will be executed via a command like {\tt dotnet run "example-2.txt"}. The program will
then be read and outputted to a MIDI file, which can be played by the user.


Output: The opening guitar riff of 2:20 by Colony House with a mock up of their drum beat and no
backing chords.


\vspace{\baselineskip}


iii. Ambient Sound
\begin{verbatim}
<song> ::= 60 bpm <melody> <beat> <chordList>
<melody> ::= e 13 beats, D5 1 beats, Bb 1 beats, G 1 beats
<beat> ::= e
<chordList> ::= G4 Bb4 D5 A6 8 beats, D4 G4 A4 8 beats, A4 C4 E4 G4 B5 8 beats
\end{verbatim}
This will be executed via a command like {\tt dotnet run "example-3.txt"}. The program will
then be read and outputted to a MIDI file, which can be played by the user.


Output: Creates an ambient background sound that can be looped and kept as a background noise or
turned into the backing of a lofi song.


\subsection{Language Concepts}


A user needs to understand basic musical theory to write programs. They also need basic
knowledge of how to write a computer program. If they do not have either of these, they still
can likely write them, but there is a large chance they won't sound too good. In terms of
primitives, users will need to know what different percussive sounds are, including both notes
(which are percussive when played on a piano) and drum sounds.


Chords are a crucial combining form to understand. Chords are made up of a collection of
three or more notes, and can be adjusted to add 2nds, 4ths, 7ths, 9ths, 11ths, and 13ths,
depending on a user's wishes. A typical chord consists of a root, a third, and a fifth. A beat
is created by repeating different percussive sounds at regular intervals on a meter. This serves
as the heart of the inputted song. Finally melody and chords are combined by overlaying the
tracks of notes on a meter and tempo given by the user.




\subsection{Formal Syntax}


\begin{verbatim}
   <num> ::= non-negative integer
   <tempo> ::= <num> bpm
   <pitch> ::= <char> <char> <num>
   <duration> ::= <num> beats
   <note> ::= <pitch> <duration>
   <melody> ::= <note>*
   <sound> ::= <num>
   <percuss> ::= <sound> <num>+
   <chord> ::= <pitch>+ <duration>
   <song> ::= <tempo> <melody> <beat> <chordList>
\end{verbatim}


\subsection{Semantics}


Our program is represented by the following components:
\begin{verbatim}
   type pitch = char * char * int
   type note = pitch * int
   type percuss = int * int list
   type chord = pitch list * int
   type song = int * note list * percuss list * chord list
\end{verbatim}


Syntax: {\tt pitch}


Abstract Syntax: {\tt pitch} of {\tt char * char * int}


Type: {\tt tuple}


Prec./Assoc.: n/a


Meaning: {\tt pitch} is a combining form that tells which note (A-G), any accidentals
(s for sharp, b for flat, and n for natural), and the octave of the given note (0-8).


\vspace{\baselineskip}
\vspace{\baselineskip}


Syntax: {\tt note}


Abstract Syntax: {\tt note} of {\tt pitch * int}


Type: {\tt tuple}


Prec./Assoc.: n/a


Meaning: {\tt note} is a combining form that gives a certain pitch and tells how long the
pitch should be held out for.


\vspace{\baselineskip}
\vspace{\baselineskip}


Syntax: {\tt percuss}


Abstract Syntax: {\tt percuss} of {\tt int * int list}


Type: {\tt tuple}


Prec./Assoc.: n/a


Meaning: {\tt percuss} is a combining form that dictates a percussive sound, where sounds are
enumerated and which beats it should be played on. These form the heart of the beat.


\vspace{\baselineskip}
\vspace{\baselineskip}


Syntax: {\tt chord}


Abstract Syntax: {\tt chord} of {\tt pitch list * int}


Type: {\tt tuple}


Prec./Assoc.: n/a


Meaning: {\tt chord} is a combining form that dictates the chord pitches and the duration of
the chord.


\vspace{\baselineskip}
\vspace{\baselineskip}


Syntax: {\tt song}


Abstract Syntax: {\tt song} of {\tt int * note list * percuss list * chord list}


Type: {\tt tuple}


Prec./Assoc.: n/a


Meaning: {\tt song} is a combining form that holds all the necessary data for a song. It
combines the tempo, melody, beat, and chords.




\vspace{\baselineskip}


\subsection{Remaining Work}
We are figuring out how to set the Tempo and debug the error which creates the corrupted midi file.

\vspace{\baselineskip}


The examples are included within the code folder.


\end{document}