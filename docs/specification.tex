\documentclass[10pt]{article}

% Lines beginning with the percent sign are comments
% This file has been commented to help you understand more about LaTeX

% DO NOT EDIT THE LINES BETWEEN THE TWO LONG HORIZONTAL LINES

%---------------------------------------------------------------------------------------------------------

% Packages add extra functionality.
\usepackage{times,graphicx,epstopdf,fancyhdr,amsfonts,amsthm,amsmath,algorithm,algorithmic,xspace,hyperref}
\usepackage[left=1in,top=1in,right=1in,bottom=1in]{geometry}
\usepackage{sect sty}	%For centering section headings
\usepackage{enumerate}	%Allows more labeling options for enumerate environments 
\usepackage{epsfig}
\usepackage[space]{grffile}
\usepackage{booktabs}
\usepackage{forest}
\usepackage[T1]{fontenc}
\usepackage[utf8]{inputenc}
\usepackage{longtable}

% This will set LaTeX to look for figures in the same directory as the .tex file
\graphicspath{.} % The dot means current directory.

\pagestyle{fancy}

\lhead{Final Project}
\rhead{\today}
\lfoot{CSCI 334: Principles of Programming Languages}
\cfoot{\thepage}
\rfoot{Fall 2023}

% Some commands for changing header and footer format
\renewcommand{\headrulewidth}{0.4pt}
\renewcommand{\headwidth}{\textwidth}
\renewcommand{\footrulewidth}{0.4pt}

% These let you use common environments
\newtheorem{claim}{Claim}
\newtheorem{definition}{Definition}
\newtheorem{theorem}{Theorem}
\newtheorem{lemma}{Lemma}
\newtheorem{observation}{Observation}
\newtheorem{question}{Question}

\setlength{\parindent}{0cm}

\begin{document}
  
\section*{AlgoRhythm Documentation}

Elias Devadoss, Himal Pandey

\subsection{Introduction}

    Our language solves the problem of having to generate and record a song when you have an idea 
you want to implement. Instead, only the pieces to the song need to be given, and the language can 
output a fully realized song based on the program. While in the modern day we can have AI generate 
songs for us, there is still some difficulty communicating exactly what you want to a chatbot or 
similar service, as well as a lack of personal creativity in having a computer generate a song. 
On the flip side, being able to think of a song from nothing can be quite challenging and many 
do not know where to start.

    This language serves as a jumping board for song ideas, where rhythm and melody can be 
mechanically reproduced and realized, without needing instruments on hand or physical skill. This 
specific language is also vital, as there are music production services where users can input 
exactly what they want to hear and receive an output. However, this language allows for a more 
loosely defined “song” to be created where a user only needs a relatively simple collection of 
chords, rhythm, and structure to be able to set the background for a song they are creating and 
test the waters. Finally, this language lowers the barriers for those who are not musically 
talented or inclined by making it relatively simple and easy to make a song, even for those 
with no experience or talent in the matter.

\subsection{Design Principles}

This language aims to be one that is simple and neat. The aesthetic will be one that aims for 
clarity and neatness. Due to the nature of how many specifics may have to be given, programs 
may be somewhat lengthy. However, they will be simple and easy to read. The idea is to have the 
language be one that follows the general principles of the language being easy to use and not a 
barrier for those who are not musically proficient. Therefore, the language will also be one 
that does not require learning much syntax or specifics, so that it is easy to implement.

\subsection{Examples}

i. Twinkle, Twinkle Little Star
\begin{verbatim}
<song> ::= 125 bpm 4/4 <melody> <beat> <chordList>
<melody> ::= C4 1 beats, C4 1 beats, G4 1 beats, G4 1 beats, A4 1 beats,
             A4 1 beats,G4 2 beats
<beat> ::= kick 1, snare 3, hi-hat 1 2 3 4
<chordList> ::= C4 E4 G2 2 beats, C4 E4 G2 2 beats, F4 A5 C5 2 beats,
                C4 E4 G2 2 beats
\end{verbatim}
This will be executed via a command like {\tt dotnet run "example-1.txt"}. The program will 
then be read and outputted to a MIDI file, which can be played by the user.

Output: Twinkle, twinkle little star with backing chords of C, C, F, C and a basic beat behind.

\vspace{\baselineskip}

ii. 2:20
\begin{verbatim}
<song> ::= 65 bpm 4/4 <melody> <beat> <chordList>
<melody> ::= e 2.75 beats, A3 0.25 beats, C4 0.25 beats, D4 0.25 beats,
             F4 0.25 beats, A3 0.25 beats, C4 0.5 beats, D4 0.5 beats,
             e 0.5 beats, D3 1 beats, e 0.25 beats
<beat> ::= kick 1 1.5, snare 3, crash 1, ride 2 3 4
<chordList> ::= e
\end{verbatim}
This will be executed via a command like {\tt dotnet run "example-2.txt"}. The program will 
then be read and outputted to a MIDI file, which can be played by the user.

Output: The opening guitar riff of 2:20 by Colony House with a mock up of their drum beat and no 
backing chords.

\vspace{\baselineskip}

iii. Ambient Sound
\begin{verbatim}
<song> ::= 60 bpm 4/4 <melody> <beat> <chordList>
<melody> ::= e 13 beats, D5 1 beats, Bb 1 beats, G 1 beats
<beat> ::= e
<chordList> ::= G4 Bb4 D5 A6 8 beats, D4 G4 A4 8 beats, A4 C4 E4 G4 B5 8 beats
\end{verbatim}
This will be executed via a command like {\tt dotnet run "example-3.txt"}. The program will 
then be read and outputted to a MIDI file, which can be played by the user.

Output: Creates an ambient background sound that can be looped and kept as a background noise or 
turned into the backing of a lofi song. 

\subsection{Language Concepts}

A user needs to understand basic musical theory to write programs. They also need basic 
knowledge of how to write a computer program. If they do not have either of these, they still 
can likely write them, but there is a large chance they won't sound too good. In terms of 
primitives, users will need to know what different percussive sounds are, including both notes 
(which are percussive when played on a piano) and drum sounds.

Chords are a crucial combining form to understand. Chords are made up of a collection of 
three or more notes, and can be adjusted to add 2nds, 4ths, 7ths, 9ths, 11ths, and 13ths, 
depending on a user's wishes. A typical chord consists of a root, a third, and a fifth. A beat 
is created by repeating different percussive sounds at regular intervals on a meter. This serves 
as the heart of the inputted song. Finally melody and chords are combined by overlaying the 
tracks of notes on a meter and tempo given by the user.


\subsection{Formal Syntax}

\begin{verbatim}
    <num> ::= positive integer
    <song> ::= <tempo> <meter> <melody> <beat> <chordList>
    <tempo> ::= <num> bpm
    <meter> ::= <num> / <noteType>
    <noteType> ::= 1 | 2 | 4 | 8 | 16 | 32
    <melody> ::= <note>*
    <note> ::= <pitch> <duration>
    <duration> ::= <num> beats
    <pitch> ::= <letter> <accidental> <octave> | e
    <letter> ::= A | B | C | D | E | F | G
    <accidental> ::= # | b | e
    <octave> ::= 0 | 1 | 2 | 3 | 4 | 5 | 6 | 7 | 8
    <beat> ::= <percuss>*
    <percuss> ::= <sound> <num>+
    <sound> ::= kick | snare | hi-hat | crash | ride | china | splash
    <chordList> ::=  <chord>*
    <chord> ::= <pitch> <pitch> <pitch>+ <duration>
\end{verbatim}

\subsection{Semantics}

Our program is represented by the following components:
\begin{verbatim}
    type song = int * meter * list note * list percuss * list chord
    type meter = int * int
    type note = pitch * int
    type pitch = char * char * int
    type percuss = string * list int
    type chord = list pitch * int
\end{verbatim}


Syntax: {\tt l}

Abstract Syntax: {\tt letter} of {\tt char}

Type: {\tt char}

Prec./Assoc.: n/a

Meaning: {\tt l} is a primitive that we represent using the {\tt char} data type. It represents 
the pitch of a given note.

\vspace{\baselineskip}
\vspace{\baselineskip}

Syntax: {\tt octave}

Abstract Syntax: {\tt num} of {\tt int}

Type: {\tt int}

Prec./Assoc.: n/a

Meaning: {\tt octave} is a primitive that we represent using the {\tt int} data type. It 
represents the octave of a given note.

\vspace{\baselineskip}
\vspace{\baselineskip}

Syntax: {\tt duration}

Abstract Syntax: {\tt num} of {\tt int}

Type: {\tt int}

Prec./Assoc.: n/a

Meaning: {\tt duration} is a primitive that we represent using the {\tt int} data type. It 
represents the number of beats a given note is held out.

\vspace{\baselineskip}
\vspace{\baselineskip}

Syntax: {\tt lo d}

Abstract Syntax: {\tt Note} of (({\tt letter} * {\tt num}) * {\tt num})

Type: {\tt char -> int -> int -> ((char * int) * int)}

Prec./Assoc.: n/a

Meaning: {\tt Note} is a combining form that represents a note. It consits of a tuple of a 
tuple of a char and an int, and an int. This represents the notes pitch, octave, and duration, 
ex. C4 2

\vspace{\baselineskip}

\subsection{Remaining Work}
We still need to implement the parsed input and generates a MIDI file. This is the majority 
of the work that we have left. We also need functions such as parseMeter and parsePerc, so that 
the user can input the meter and percussive beats in a piece. Lastly, we need to tidy up some 
odd bits and pieces, such as parseLetter, which parses any letter, whereas letters that are not 
A-G should not be counted as successful parses.

\vspace{\baselineskip}

The examples are included within the code folder.

\end{document}