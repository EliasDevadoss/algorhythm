\documentclass[10pt]{article}

% Lines beginning with the percent sign are comments
% This file has been commented to help you understand more about LaTeX

% DO NOT EDIT THE LINES BETWEEN THE TWO LONG HORIZONTAL LINES

%---------------------------------------------------------------------------------------------------------

% Packages add extra functionality.
\usepackage{times,graphicx,epstopdf,fancyhdr,amsfonts,amsthm,amsmath,algorithm,algorithmic,xspace,hyperref}
\usepackage[left=1in,top=1in,right=1in,bottom=1in]{geometry}
\usepackage{sect sty}	%For centering section headings
\usepackage{enumerate}	%Allows more labeling options for enumerate environments 
\usepackage{epsfig}
\usepackage[space]{grffile}
\usepackage{booktabs}
\usepackage{forest}
\usepackage[T1]{fontenc}
\usepackage[utf8]{inputenc}
\usepackage{longtable}

% This will set LaTeX to look for figures in the same directory as the .tex file
\graphicspath{.} % The dot means current directory.

\pagestyle{fancy}

\lhead{Final Project}
\rhead{\today}
\lfoot{CSCI 334: Principles of Programming Languages}
\cfoot{\thepage}
\rfoot{Fall 2023}

% Some commands for changing header and footer format
\renewcommand{\headrulewidth}{0.4pt}
\renewcommand{\headwidth}{\textwidth}
\renewcommand{\footrulewidth}{0.4pt}

% These let you use common environments
\newtheorem{claim}{Claim}
\newtheorem{definition}{Definition}
\newtheorem{theorem}{Theorem}
\newtheorem{lemma}{Lemma}
\newtheorem{observation}{Observation}
\newtheorem{question}{Question}

\setlength{\parindent}{0cm}

\begin{document}
  
\section*{Documentation}

Elias Devadoss, Himal Pandey

\subsection{BNF Grammar}

\begin{verbatim}
    <num> ::= positive integer
    <song> ::= <tempo> <meter> <melody> <beat> <chordList>
    <tempo> ::= <num> bpm
    <meter> ::= <num> / <noteType>
    <noteType> ::= 1 | 2 | 4 | 8 | 16 | 32
    <melody> ::= <note>*
    <note> ::= <pitch> <duration>
    <duration> ::= <num> beats
    <pitch> ::= <letter> <accidental> <octave> | e
    <letter> ::= A | B | C | D | E | F | G
    <accidental> ::= # | b | e
    <octave> ::= 0 | 1 | 2 | 3 | 4 | 5 | 6 | 7 | 8
    <beat> ::= <percuss>*
    <percuss> ::= <sound> <num>+
    <sound> ::= kick | snare | hi-hat | crash | ride | china | splash
    <chordList> ::=  <chord>*
    <chord> ::= <pitch> <pitch> <pitch>+ <duration>
\end{verbatim}

\subsection{Syntax}

Syntax: {\tt l}

Abstract Syntax: {\tt letter} of {\tt char}

Type: {\tt char}

Prec./Assoc.: n/a

Meaning: {\tt l} is a primitive that we represent using the {\tt char} data type. It represents 
the pitch of a given note.

\vspace{\baselineskip}
\vspace{\baselineskip}

Syntax: {\tt octave}

Abstract Syntax: {\tt num} of {\tt int}

Type: {\tt int}

Prec./Assoc.: n/a

Meaning: {\tt octave} is a primitive that we represent using the {\tt int} data type. It 
represents the octave of a given note.

\vspace{\baselineskip}
\vspace{\baselineskip}

Syntax: {\tt duration}

Abstract Syntax: {\tt num} of {\tt int}

Type: {\tt int}

Prec./Assoc.: n/a

Meaning: {\tt duration} is a primitive that we represent using the {\tt int} data type. It 
represents the number of beats a given note is held out.

\vspace{\baselineskip}
\vspace{\baselineskip}

Syntax: {\tt lo d}

Abstract Syntax: {\tt Note} of (({\tt letter} * {\tt num}) * {\tt num})

Type: {\tt char -> int -> int -> ((char * int) * int)}

Prec./Assoc.: n/a

Meaning: {\tt Note} is a combining form that represents a note. It consits of a tuple of a 
tuple of a char and an int, and an int. This represents the notes pitch, octave, and duration, 
ex. C4 2

\end{document}